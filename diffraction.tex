\input{text/diss}
\usepackage{setspace}

\begin{document}

\def\labauthors{Понур К.А., Сарафанов Ф.Г., Сидоров Д.А.}
\def\labgroup{420}
\def\labnumber{320}
\def\labtheme{Дифракций Фраунгофера}
\renewcommand{\vec}{\mathbf}
\renewcommand{\Re}{\operatorname{Re}}
\renewcommand{\Im}{\operatorname{Im}}
\renewcommand{\phi}{\varphi}
\renewcommand{\kappa}{\varkappa}
\renewcommand{\hat}{\widehat}
%%%%%%%%%%%%%%%%%%%%%%%%%%%%%%%%%%%%%%%%%%%%%%%%%%%%%%%%%%%%%%%%%%%%%%%%%%%%%%%
\input{text/titlepage}
%%%%%%%%%%%%%%%%%%%%%%%%%%%%%%%%%%%%%%%%%%%%%%%%%%%%%%%%%%%%%%%%%%%%%%%%%%%%%%%
\begin{spacing}{1}
\tableofcontents
\end{spacing}
% \setstretch{1.2}
\newpage
%%%%%%%%%%%%%%%%%%%%%%%%%%%%%%%%%%%%%%%%%%%%%%%%%%%%%%%%%%%%%%%%%%%%%%%%%%%%%%%
 \section{Теоретическая часть}
В данной работе узучается дифракция на следующих объектах: 1) на одной щели, 2) на двух щелях, 3) на решетке ищ нескольких щелей. Наблюдения и измерения производятся при помощи гониометра -- оптического прибора, предназначенного для измерения углов с большой точностью. 

При помощи гониометра изучают угловое распределение интенсивности дифрагированного света. Углы дифракции изменяются оптическим компенсатором (микроскопом с отсчетным микрометром).

При дифракции Фраунгофера на щели интенсивность излучения в плоскости $xy$, перпендикулярной щели, зависит от угла дифракции по закону
\begin{equation}
	I_{\theta}=I_0\frac{\sin^2\frac{kb\sin\theta}{2}}{(\frac{kb\sin\theta}{2})^2},		
\end{equation}
где $I_0$- интенсивность в направлении $\theta=0$, $I_{\theta}$- интенсивность в направлении $\theta$, $b$- ширина щели, $k$- волновое число.

При дифракции Фраунгофера от решетки с периодом $d$ из $N$ одинаковых щелец ширины $b$ зависимость интенсивность $I_{\theta}$ описывается формулой
\begin{equation}
	I_{\theta}=I_0\frac{\sin^2\frac{kb\sin\theta}{2}}{(\frac{kb\sin\theta}{2})^2}
	\cdot
	\frac{\sin^2\frac{Nkd\sin{\theta}}{2}}{\sin^2\frac{kd\sin{\theta}}{2}}	
\end{equation}

Рассмотрим влияние размеров источника света на вид дифракционной картины при дифрауции на двух щелях. В данной работе источником света служит щель коллиматора. Обозначим ширину этой щели $l$, а её угловой размер $\alpha$. %Нужен рис.3!
От каждой точки источника на объект дифракции падает плоская волна и создает в фокальной плоскости дифракционную картину. Крайние точки источника $K$ и $f$ создают картины, центры которых $K'$ и $f'$ смещены относительно друг друга на угловое расстояние $\alpha$. %Рис 3!

Контрастность дифракционных картин характеризуется видимостью
\begin{equation}
	V=\frac{I_{max}-I_{min}}{I_{max}+I_{min}},
\end{equation}
где $I_{max}$- интенсивность в максимуме, $I_{min}$- интенсивность в ближайшем к нему минимуме.

Видимость дифракционной картины от двух щелей зависит от углового размера источника $\alpha$. Если яркость источника одинакова по всей ширине, то при увеличении $\alpha$ первый минимум вилимостти наступит, когда $\alpha$ станет равно $\theta_1$- угловому расстоянию между нелевым и первым максимами. При малых углах
\begin{equation}
	\sin{\theta_1}\simeq\theta_1=\frac{\lambda}{d},\; \alpha=\frac{l}{F}
\end{equation}
здесь $\lambda$- длина световой волны источника, $d$- фокусное расстояние между щелями на экране, $F$- фокусное расстояние линзы коллиматора.

Условие первого минимума имеет вид
\begin{equation}
	\label{eq:3}
	l=\theta_1F=\frac{\lambda F}{d}
\end{equation}
Формула (\ref{eq:3}) даёт возможность определить шишину источника света по найденному опытным путём расстоянию $d$ между щелями, при котором наступает размытие дифракционной картины.

Таким был метод, использованный в 1920 г. Майкельсоном для измерения углового расстояния между компонентами двойной звезды Капеллы и диаметра звезды Бетельгейзе.
 %%%%%%%%%%%%%%%%%%%%%%%%%%%%%%%%%%%%%%%%%%%%%%%%%%%%%%%%%%%%%%%%%%%%%%%%%%%%%%%
\newpage
\section{Заключение}

\end{document}