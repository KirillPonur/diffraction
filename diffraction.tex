\input{text/diss}
\usepackage{setspace}
\usepackage{amsmath}

\DeclareMathOperator{\sinc}{sinc}
\newcommand{\dif}[3]{


\pgfplotstablegetelem{0}{#2}\of#1

% add column LocalDistance
\pgfplotstablecreatecol
    [expr={\thisrow{#2} - \pgfplotsretval}]
    {LocalDistance#3}{#1}

% add column DifferenceDistance
\pgfplotstablecreatecol
    % [expr={-\thisrow{LocalDistance} + \prevrow{LocalDistance}}]
    % [expr={rad(180)}]
    [expr={-\thisrow{LocalDistance#3} + \prevrow{LocalDistance#3}}]
    {#3}{#1}

}
\newcommand{\Exp}[1]{
	\exp\left(#1\right)
}
\newcommand{\Sinc}[1]{
	\sinc\left(#1\right)
}
\newcommand{\Sin}[1]{
	\sin\left(#1\right)
}
\begin{document}

\def\labauthors{Понур К.А., Сарафанов Ф.Г., Сидоров Д.А.}
\def\labgroup{420}
\def\labnumber{320}
\def\labtheme{Дифракций Фраунгофера}
\renewcommand{\vec}{\mathbf}
\renewcommand{\Re}{\operatorname{Re}}
\renewcommand{\Im}{\operatorname{Im}}
\renewcommand{\phi}{\varphi}
\renewcommand{\kappa}{\varkappa}
\renewcommand{\hat}{\widehat}
%%%%%%%%%%%%%%%%%%%%%%%%%%%%%%%%%%%%%%%%%%%%%%%%%%%%%%%%%%%%%%%%%%%%%%%%%%%%%%%
\input{text/titlepage}
%%%%%%%%%%%%%%%%%%%%%%%%%%%%%%%%%%%%%%%%%%%%%%%%%%%%%%%%%%%%%%%%%%%%%%%%%%%%%%%
\begin{spacing}{1}
\tableofcontents
\end{spacing}
% \setstretch{1.2}
\newpage
%%%%%%%%%%%%%%%%%%%%%%%%%%%%%%%%%%%%%%%%%%%%%%%%%%%%%%%%%%%%%%%%%%%%%%%%%%%%%%%
 \section{Теоретическая часть}
В данной работе изучается дифракция на одной щели, двух щелях и на решетке из нескольких щелей. Наблюдения и измерения производятся при помощи гониометра -- оптического прибора для измерения углов с большой точностью. 

При помощи гониометра изучают угловое распределение интенсивности дифрагированного света. Углы дифракции изменяются оптическим компенсатором (микроскопом с отчетным микрометром).

При дифракции Фраунгофера на щели интенсивность излучения в плоскости $xy$, перпендикулярной щели, зависит от угла дифракции по закону
\begin{equation}
	I_{\theta}=I_0\frac{\sin^2\frac{kb\sin\theta}{2}}{(\frac{kb\sin\theta}{2})^2},		
\end{equation}
где $I_0$- интенсивность в направлении $\theta=0$, $I_{\theta}$- интенсивность в направлении $\theta$, $b$- ширина щели, $k$- волновое число.

При дифракции Фраунгофера от решетки с периодом $d$ из $N$ одинаковых щелец ширины $b$ зависимость интенсивность $I_{\theta}$ описывается формулой
\begin{equation}
	I_{\theta}=I_0\frac{\sin^2\frac{kb\sin\theta}{2}}{(\frac{kb\sin\theta}{2})^2}
	\cdot
	\frac{\sin^2\frac{Nkd\sin{\theta}}{2}}{\sin^2\frac{kd\sin{\theta}}{2}}	
\end{equation}

Рассмотрим влияние размеров источника света на вид дифракционной картины при дифракции на двух щелях. В данной работе источником света служит щель коллиматора. Обозначим ширину этой щели $l$, а её угловой размер $\alpha$. %Нужен рис.3!
От каждой точки источника на объект дифракции падает плоская волна и создает в фокальной плоскости дифракционную картину. Крайние точки источника $K$ и $f$ создают картины, центры которых $K'$ и $f'$ смещены относительно друг друга на угловое расстояние $\alpha$. %Рис 3!

Контрастность дифракционных картин характеризуется видимостью
\begin{equation}
	V=\frac{I_{max}-I_{min}}{I_{max}+I_{min}},
\end{equation}
где $I_{max}$- интенсивность в максимуме, $I_{min}$- интенсивность в ближайшем к нему минимуме.

Видимость дифракционной картины от двух щелей зависит от углового размера источника $\alpha$. Если яркость источника одинакова по всей ширине, то при увеличении $\alpha$ первый минимум видимости наступит, когда $\alpha$ станет равно $\theta_1$- угловому расстоянию между нулевым и первым максимами. При малых углах
\begin{equation}
	\sin{\theta_1}\simeq\theta_1=\frac{\lambda}{d},\; \alpha=\frac{l}{F}
\end{equation}
здесь $\lambda$- длина световой волны источника, $d$- фокусное расстояние между щелями на экране, $F$- фокусное расстояние линзы коллиматора.

Условие первого минимума имеет вид
\begin{equation}
	\label{eq:3}
	l=\theta_1F=\frac{\lambda F}{d}
\end{equation}
Формула (\ref{eq:3}) даёт возможность определить ширину источника света по найденному опытным путём расстоянию $d$ между щелями, при котором наступает размытие дифракционной картины.

Таким был метод, использованный в 1920 г. Майкельсоном для измерения углового расстояния между компонентами двойной звезды Капеллы и диаметра звезды Бетельгейзе.

\subsection{Вывод уравнения интенсивностей при дифракции Фраунгофера на решетке}

\begin{figure}[H]
	\centering
	\includegraphics[]{ris/diff}
	\caption{Caption here}
	\label{fig:figure1}
\end{figure}

Сначала выведем дифракцию на первой щели, пользуясь принципом Гюйгенса-Френеля. 

Пусть на щель падает свет амплитудой $E_0$, длиной волны $\lambda$.  Щель разобьем на бесконечно малые излучатели шириной $dx$ и с амплитудой излучаемой волны $\frac{E_0}{b}dx$.  

Набег фазы для каждого такого излучателя относительно излучателя с координатой $x=0$ будет $k\Delta=k\cdot x \sin\Theta$:

\begin{equation}
	d\hat{E}(x)=\frac{\hat{E}_0}{b}\cdot\Exp{i\cdot k x \sin\Theta} dx
\end{equation}
Проинтегрируем по всей щели:
\begin{gather}
	\hat{E}_{1}=\hat{E}_0\int\limits_0^b \frac{1}{i\cdot kb\sin\Theta} \Exp{i\cdot k x \sin\Theta} d[i\cdot k x \sin\Theta]=\\=
	\hat{E}_0 \frac{\Exp{i\cdot k b \sin\Theta}-1}{i\cdot kb\sin\Theta}=
	\hat{E}_0\Exp{i\cdot \frac{k b \sin\Theta}{2}} \frac{\Exp{i\cdot \frac{k b \sin\Theta}{2}}-\Exp{-i\cdot \frac{k b \sin\Theta}{2}}}{i\cdot kb\sin\Theta}=\\=
	\hat{E}_0\Exp{i\cdot \frac{k b \sin\Theta}{2}} \Sinc{\frac{k b \sin\Theta}{2}}
\end{gather}

<<Спрячем>> экспоненту в комплексную амплитуду. Это не повлияет на решение, так как для всех щелей набег фазы в этой экспоненте будет одинаков.

\begin{equation}
	\hat{E}_{1}=\hat{E}_{a}\Sinc{\frac{k b \sin\Theta}{2}}
\end{equation}

Теперь рассмотрим сложение волн, пришедших от всех щелей в дифракционной решетке. Нетрудно показать, что набег фазы будет зависеть от номера щели и угла $\Theta$:
\begin{equation}
	\hat{E}_m=\hat{E}_{1}\Exp{i\cdot k (m-1)d\sin\Theta},
\end{equation}
где $m$ -- номер щели.

Тогда можем записать сумму волн:
\begin{equation}
	\hat{E}(\Theta)=\hat{E}_1 \left(
		1+\Exp{i\cdot kd\sin\Theta}+\ldots+\Exp{i\cdot k (N-1)d\sin\Theta}
	\right)
\end{equation}

Второй множитель здесь -- решеточный множитель, который дает постоянный сдвиг фазы и множитель вида $\sin  Nx / \sin x$. Нетрудно показать, что тогда

\begin{equation}
	\hat{E}(\Theta) \sim \hat{E}_1 \Sinc{\frac{k b \sin\Theta}{2}}
	\left[
		\frac{\Sin{\frac{Nk d \sin\Theta}{2}}}{\Sin{\frac{k d \sin\Theta}{2}}}
	\right]
\end{equation}

И тогда окончательный результат:

\begin{equation}
	\label{eq:intensity}
	I(\Theta) = I_0 \sinc^2\left(\frac{k b \sin\Theta}{2}\right)
	\left[
		\frac{\displaystyle\Sin{\frac{Nk d \sin\Theta}{2}}}{\displaystyle\Sin{\frac{k d \sin\Theta}{2}}}
	\right]^2
\end{equation}

 %%%%%%%%%%%%%%%%%%%%%%%%%%%%%%%%%%%%%%%%%%%%%%%%%%%%%%%%%%%%%%%%%%%%%%%%%%%%%%%
\subsection{Вывод условия первого минимума видимости}
Полосы на экране будут видны достаточно отчётливо, пока расстояние между полосами $\Delta x$ будет меньше $\delta x$.
$\delta x$ связана с линейным размером источника $l$ соотношением
\begin{equation}
	\delta x=\frac{dl}{F}
\end{equation}
Угловой размер источника:
\begin{equation}
	\alpha=\frac{l}{F}
\end{equation}
Угловая ширина полос:
\begin{equation}
	\theta=\frac{\lambda}{d}
\end{equation}
Тогда при $\alpha<\theta$ картина будет видна достаточно отчетливо.
Отсюда получаем условие первого минимума видимости
\begin{equation}
	l=\frac{\lambda F}{d}
\end{equation}




\newpage
\section{Результаты эксперимента}

\subsection{Качественные наблюдения}
\subsubsection{Условия эксперимента}
Изначально свет идет от лампочки накаливания, размер спиральки которой 3 мм.

\subsubsection{Изменение $b$}
С изменением ширины щели решетки -- уменьшением $b$ картинка расширяется, увеличивается расстояние между максимумами

\subsubsection{Изменение $d$}
Экспериментально было установлено, что с изменением периода решетки (уменьшением $d$) картинка расширяется, увеличивается расстояние между максимумами

Теоретически это нетрудно обосновать. Рассмотрим решёточный множитель в формуле (\ref{eq:intensity}). Функция имеет минимумы в точках 
\begin{equation}
  \sin{\theta_m}=\frac{\lambda m}{Nd},\: m=1,2\dots\frac{Nd}{\lambda}.
\end{equation}

Таким образом, при уменьшении d увеличивается расстояние между максимумами.

\subsubsection{Поворот дифракционной решётки}
С увеличением угла, под которым расположена дифракционная решетка картина расширяется


\subsubsection{Изменение $\lambda$}
Для красного ширина центрального максимума шире, чем для зеленого.
Полушириной центрального максимума будем называть угловое расстояние от $\theta=0$ до ближайшего минимума.
Тогда
\begin{equation}
 	\theta_0=\arcsin\frac{\lambda}{Nd}
 \end{equation} 
 То есть при увеличении длины волны картинка расширяется. Что мы и наблюдали в эксперименте.

\subsubsection{Изменение длины щели источника}
Дифракционная картина при изменении длины щели источника не изменяется. 

\subsubsection{Изменение ширины щели источника}

\begin{table}[H]
	    \caption{Показания микрометра щели источника и ширина щели для разных дифракционных картин: З--щель закрыта, Ч--чёткая дифракционная картина, Р--размытая дифракционная картина}
	    \label{tab:chem1}
	    % \pgfkeys{/pgf/number format/.cd,
		% fixed,  1000 sep={\,}}

	\pgfplotstableread{data/dx.tsv}\mytable
	\pgfplotstableset{
	% multicolumn names, % allows to have multicolumn names
	% header=has colnames,
	dec sep align,
	col sep=tab, % the seperator in our .csv file
	fixed zerofill, 
	% precision=4,			
	empty cells with={\textbf{--}},
	every head row/.style={
	before row={\toprule},
	after row={
		\midrule}
		},
	columns/T/.style={
		column name={З, $z$, мм$\cdot10^{-2}$},
		precision=0		
	},	
	columns/CH/.style={
		column name={Ч, $z$, мм$\cdot10^{-2}$},
		precision=0		
	},	
	columns/R/.style={
		column name={Р, $z$, мм$\cdot10^{-2}$},
		precision=0		
	},
	columns/R2/.style={
		column name={Р, $\Delta x$, мм$\cdot10^{-2}$},
		precision=0		
	},	
	columns/CH2/.style={
		column name={Ч, $\Delta x$, мм$\cdot10^{-2}$},
		precision=0		
	},		
	every last row/.style={after row=\bottomrule},
	every row/.style={after row=\midrule}, 
	% columns={N, deg,min,sec},	
	columns={T,CH,R,CH2,R2}	
	% dec zerofill
	% fixed,fixed zerofill,
	% precision=3
	% every even column/.style={
	% 	% column type/.add={>{\columncolor[gray]{.8}}}{}
	% },
	% every even row/.style={
	% 	before row={\rowcolor[gray]{0.95}}
	% },	
	}
	\centering
	\pgfplotstabletypeset[]{\mytable}
\end{table}

\subsubsection{Порядок следования цветов}
Распределение цветов при дифракции в белом свете: ЗЖК


\subsection{Дифракционные картины для разных решёток}
\subsubsection{Дифракция на одной щели}
\begin{table}[H]
	    \caption{$b=0.52$ мм, $N=1$, по минимумам}
	    \label{tab:chem1}
	    % \pgfkeys{/pgf/number format/.cd,
		% fixed,  1000 sep={\,}}

	\pgfplotstableread{data/N1.tsv}\mytable
	
\dif{\mytable}{deg}{deg2}
\dif{\mytable}{min}{min2}
\dif{\mytable}{sec}{sec2}

\pgfplotstablecreatecol
    % [expr={-\thisrow{LocalDistance} + \prevrow{LocalDistance}}]
    % [expr={rad(180)}]
    % [expr={rad(\thisrow{deg2}+1/60*\thisrow{min2}+1/3600*\thisrow{sec2})*1000}]
    [expr={3600*\thisrow{deg2}+60*\thisrow{min2}+\thisrow{sec2}}]
    {deltas}{\mytable}
\pgfplotstablecreatecol
    % [expr={-\thisrow{LocalDistance} + \prevrow{LocalDistance}}]
    % [expr={rad(180)}]
    % [expr={rad(\thisrow{deg2}+1/60*\thisrow{min2}+1/3600*\thisrow{sec2})*1000}]
    [expr={\thisrow{deltas}/8}]
    {ddeltas}{\mytable}
% \pgfplotstabletypeset
%     [columns={N,deg,min,sec,deg2, min2,sec2,deltas}]
%     {\mytable}
\pgfplotstableset{
	% multicolumn names, % allows to have multicolumn names
	% header=has colnames,
	dec sep align,
	col sep=tab, % the seperator in our .csv file
	fixed zerofill, 
	% precision=4,			
	empty cells with={\textbf{--}},
	every head row/.style={
	before row={\toprule},
	after row={
		\midrule}
		},
	columns/N/.style={
		column name={N},
		precision=0		
	},	
	columns/deg/.style={
		column name={$\Theta^\circ$},
		precision=0		
	},	
	columns/min/.style={
		column name={$\Theta'$},
		precision=0		
	},	
	columns/sec/.style={
		column name={$\Theta''$},
		precision=0,
		column type/.add={}{|}
	},	
	columns/sec2/.style={
		column name={$\Delta\Theta''$},
		precision=0,
		column type/.add={}{|},
	},	
	columns/min2/.style={
		column name={$\Delta\Theta'$},
		precision=0		
	},	
	columns/deg2/.style={
		column name={$\Delta\Theta^\circ$},
		precision=0,
	},	
	columns/deltas/.style={
		column name={$\Delta\Theta$, $''$},
		precision=0,
		dec sep align,
		% column type = {r}
	},	
	columns/ddeltas/.style={
		column name={погрешность, $''$},
		precision=0,
		dec sep align,
		% column type = {r}
	},	
	every last row/.style={after row=\bottomrule},
	every row/.style={after row=\midrule}, 
	% columns={N, deg,min,sec},	
	columns={N,deg,min,sec,deg2, min2,sec2,deltas, ddeltas}	
	% dec zerofill
	% fixed,fixed zerofill,
	% precision=3
	% every even column/.style={
	% 	% column type/.add={>{\columncolor[gray]{.8}}}{}
	% },
	% every even row/.style={
	% 	before row={\rowcolor[gray]{0.95}}
	% },	
	}
	\centering
	\pgfplotstabletypeset[]{\mytable}
\end{table}
\begin{figure}[H]
	\centering
	\includegraphics[]{plot/N1}
	\caption{Теоретический вид распределения интенсивности, дифракция на одной щели}
	\label{fig:figure1}
\end{figure}
\subsubsection{Дифракция на двух щелях}

\begin{table}[H]
	    \caption{$b=0.52$ мм, $d=1.5$ мм, $N=2$, по минимумам}
	    \label{tab:chem1} 
	    % \pgfkeys{/pgf/number format/.cd,
		% fixed,  1000 sep={\,}}

	\pgfplotstableread{data/N2.tsv}\mytable
	
\dif{\mytable}{deg}{deg2}
\dif{\mytable}{min}{min2}
\dif{\mytable}{sec}{sec2}

\pgfplotstablecreatecol
    % [expr={-\thisrow{LocalDistance} + \prevrow{LocalDistance}}]
    % [expr={rad(180)}]
    % [expr={rad(\thisrow{deg2}+1/60*\thisrow{min2}+1/3600*\thisrow{sec2})*1000}]
    [expr={3600*\thisrow{deg2}+60*\thisrow{min2}+\thisrow{sec2}}]
    {deltas}{\mytable}
\pgfplotstablecreatecol
    % [expr={-\thisrow{LocalDistance} + \prevrow{LocalDistance}}]
    % [expr={rad(180)}]
    % [expr={rad(\thisrow{deg2}+1/60*\thisrow{min2}+1/3600*\thisrow{sec2})*1000}]
    [expr={\thisrow{deltas}/8}]
    {ddeltas}{\mytable}
% \pgfplotstabletypeset
%     [columns={N,deg,min,sec,deg2, min2,sec2,deltas}]
%     {\mytable}
\pgfplotstableset{
	% multicolumn names, % allows to have multicolumn names
	% header=has colnames,
	dec sep align,
	col sep=tab, % the seperator in our .csv file
	fixed zerofill, 
	% precision=4,			
	empty cells with={\textbf{--}},
	every head row/.style={
	before row={\toprule},
	after row={
		\midrule}
		},
	columns/N/.style={
		column name={N},
		precision=0		
	},	
	columns/deg/.style={
		column name={$\Theta^\circ$},
		precision=0		
	},	
	columns/min/.style={
		column name={$\Theta'$},
		precision=0		
	},	
	columns/sec/.style={
		column name={$\Theta''$},
		precision=0,
		column type/.add={}{|}
	},	
	columns/sec2/.style={
		column name={$\Delta\Theta''$},
		precision=0,
		column type/.add={}{|},
	},	
	columns/min2/.style={
		column name={$\Delta\Theta'$},
		precision=0		
	},	
	columns/deg2/.style={
		column name={$\Delta\Theta^\circ$},
		precision=0,
	},	
	columns/deltas/.style={
		column name={$\Delta\Theta$, $''$},
		precision=0,
		dec sep align,
		% column type = {r}
	},	
	columns/ddeltas/.style={
		column name={погрешность, $''$},
		precision=0,
		dec sep align,
		% column type = {r}
	},	
	every last row/.style={after row=\bottomrule},
	every row/.style={after row=\midrule}, 
	% columns={N, deg,min,sec},	
	columns={N,deg,min,sec,deg2, min2,sec2,deltas, ddeltas}	
	% dec zerofill
	% fixed,fixed zerofill,
	% precision=3
	% every even column/.style={
	% 	% column type/.add={>{\columncolor[gray]{.8}}}{}
	% },
	% every even row/.style={
	% 	before row={\rowcolor[gray]{0.95}}
	% },	
	}
	\centering
	\pgfplotstabletypeset[]{\mytable}
\end{table}
\begin{figure}[H]
	\centering
	\includegraphics[]{plot/N2}
	\caption{Теоретический вид распределения интенсивности, дифракция на двух щелях}
	\label{fig:figure1}
\end{figure}
\subsubsection{Дифракция на пятнадцати щелях}
\begin{table}[H]
	    \caption{$b=1$ мм, $d=2$ мм, $N=15$, по максимумам}
	    \label{tab:chem1}
	    % \pgfkeys{/pgf/number format/.cd,
		% fixed,  1000 sep={\,}}

	\pgfplotstableread{data/N15.tsv}\mytable
	
\dif{\mytable}{deg}{deg2}
\dif{\mytable}{min}{min2}
\dif{\mytable}{sec}{sec2}

\pgfplotstablecreatecol
    % [expr={-\thisrow{LocalDistance} + \prevrow{LocalDistance}}]
    % [expr={rad(180)}]
    % [expr={rad(\thisrow{deg2}+1/60*\thisrow{min2}+1/3600*\thisrow{sec2})*1000}]
    [expr={3600*\thisrow{deg2}+60*\thisrow{min2}+\thisrow{sec2}}]
    {deltas}{\mytable}
\pgfplotstablecreatecol
    % [expr={-\thisrow{LocalDistance} + \prevrow{LocalDistance}}]
    % [expr={rad(180)}]
    % [expr={rad(\thisrow{deg2}+1/60*\thisrow{min2}+1/3600*\thisrow{sec2})*1000}]
    [expr={\thisrow{deltas}/8}]
    {ddeltas}{\mytable}
% \pgfplotstabletypeset
%     [columns={N,deg,min,sec,deg2, min2,sec2,deltas}]
%     {\mytable}
\pgfplotstableset{
	% multicolumn names, % allows to have multicolumn names
	% header=has colnames,
	dec sep align,
	col sep=tab, % the seperator in our .csv file
	fixed zerofill, 
	% precision=4,			
	empty cells with={\textbf{--}},
	every head row/.style={
	before row={\toprule},
	after row={
		\midrule}
		},
	columns/N/.style={
		column name={N},
		precision=0		
	},	
	columns/deg/.style={
		column name={$\Theta^\circ$},
		precision=0		
	},	
	columns/min/.style={
		column name={$\Theta'$},
		precision=0		
	},	
	columns/sec/.style={
		column name={$\Theta''$},
		precision=0,
		column type/.add={}{|}
	},	
	columns/sec2/.style={
		column name={$\Delta\Theta''$},
		precision=0,
		column type/.add={}{|},
	},	
	columns/min2/.style={
		column name={$\Delta\Theta'$},
		precision=0		
	},	
	columns/deg2/.style={
		column name={$\Delta\Theta^\circ$},
		precision=0,
	},	
	columns/deltas/.style={
		column name={$\Delta\Theta$, $''$},
		precision=0,
		dec sep align,
		% column type = {r}
	},	
	columns/ddeltas/.style={
		column name={погрешность, $''$},
		precision=0,
		dec sep align,
		% column type = {r}
	},	
	every last row/.style={after row=\bottomrule},
	every row/.style={after row=\midrule}, 
	% columns={N, deg,min,sec},	
	columns={N,deg,min,sec,deg2, min2,sec2,deltas, ddeltas}	
	% dec zerofill
	% fixed,fixed zerofill,
	% precision=3
	% every even column/.style={
	% 	% column type/.add={>{\columncolor[gray]{.8}}}{}
	% },
	% every even row/.style={
	% 	before row={\rowcolor[gray]{0.95}}
	% },	
	}
	\centering
	\pgfplotstabletypeset[]{\mytable}
\end{table}
\begin{figure}[H]
	\centering
	\includegraphics[]{plot/N15}
	\caption{Теоретический вид распределения интенсивности, дифракция на пятнадцати щелях}
	\label{fig:figure1}
\end{figure}



\end{document}